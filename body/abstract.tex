%%==================================================
%% abstract.tex for SJTU Master Thesis
%% based on CASthesis
%% modified by wei.jianwen@gmail.com
%% version: 0.3a
%% Encoding: UTF-8
%% last update: Dec 5th, 2010
%%==================================================

\begin{abstract}

	随着计算机编程语言特性的发展,在程序分析领域,静态分析方法受到了越来越多的约束,动态分析的重要性越来越高。动态程序,顾名思义就是在执行程序的同时,对程序进行分析,以达到检测其中的bug、对其进行优化等等目的。相比于静态分析,动态分析能有效地减少分析时间,并能在不获得源代码的情况下开始进行分析。

	然而在动态分析中,一些重量级的动态分析程序所产生的问题也对研究者提出了挑战。一方面,这些分析程序的运行开销较大,给原程序的执行带来了很大的影响,降低了原程序的执行效率;另一方面,由于分析程序是伴随着被分析程序的执行而执行的,其执行顺序、执行线程等都受到了限制,有些能进行并行或者乱序执行的方法没有得到优化,使得执行过程过于冗长。而计算机技术的提高,多核、多处理器计算机的出现和发展,给并行执行动态程序分析带来了可能。

	相关研究发现,在影响效率的动态分析程序中,有一部分代码与方法并不会对原程序产生影响,这就意味着这些代码可以被剥离原程序进行并行执行,从而提高其效率,减少其执行时间。

	但简单地对这些程序进行并行化可能会引入负面的影响。一方面,由于某一些动态程序的方法需要访问一些共用数据段,仅仅不采取策略而只是单纯地将这些程序分配给不同的核执行,可能会造成更多不必要的访问共享数据结构的开销,从而使总体效率不升反降。

	一个有效的解决方案就是把需要并行执行的动态程序先放入缓存中,在累积到一定数量的方法时,将该缓存统一执行,这样一方面提高了数据的局部性,另一方面也减少了对共享数据结构频繁访问所产生的开销。本课题即是基于这个思想所产生。

  \keywords{\large 并行计算 \quad 程序分析 \quad 代码生成}
\end{abstract}

\begin{englishabstract}

An imperial edict issued in 1896 by Emperor Guangxu, established Nanyang Public School in Shanghai. The normal school, school of foreign studies, middle school and a high school were established. Sheng Xuanhuai, the person responsible for proposing the idea to the emperor, became the first president and is regarded as the founder of the university.

During the 1930s, the university gained a reputation of nurturing top engineers. After the foundation of People's Republic, some faculties were transferred to other universities. A significant amount of its faculty were sent in 1956, by the national government, to Xi'an to help build up Xi'an Jiao Tong University in western China. Afterwards, the school was officially renamed Shanghai Jiao Tong University.

Since the reform and opening up policy in China, SJTU has taken the lead in management reform of institutions for higher education, regaining its vigor and vitality with an unprecedented momentum of growth. SJTU includes five beautiful campuses, Xuhui, Minhang, Luwan Qibao, and Fahua, taking up an area of about 3,225,833 m2. A number of disciplines have been advancing towards the top echelon internationally, and a batch of burgeoning branches of learning have taken an important position domestically.

Today SJTU has 31 schools (departments), 63 undergraduate programs, 250 masters-degree programs, 203 Ph.D. programs, 28 post-doctorate programs, and 11 state key laboratories and national engineering research centers.

SJTU boasts a large number of famous scientists and professors, including 35 academics of the Academy of Sciences and Academy of Engineering, 95 accredited professors and chair professors of the "Cheung Kong Scholars Program" and more than 2,000 professors and associate professors.

Its total enrollment of students amounts to 35,929, of which 1,564 are international students. There are 16,802 undergraduates, and 17,563 masters and Ph.D. candidates. After more than a century of operation, Jiao Tong University has inherited the old tradition of "high starting points, solid foundation, strict requirements and extensive practice." Students from SJTU have won top prizes in various competitions, including ACM International Collegiate Programming Contest, International Mathematical Contest in Modeling and Electronics Design Contests. Famous alumni include Jiang Zemin, Lu Dingyi, Ding Guangen, Wang Daohan, Qian Xuesen, Wu Wenjun, Zou Taofen, Mao Yisheng, Cai Er, Huang Yanpei, Shao Lizi, Wang An and many more. More than 200 of the academics of the Chinese Academy of Sciences and Chinese Academy of Engineering are alumni of Jiao Tong University.

  \englishkeywords{\large SJTU, master thesis, XeTeX/LaTeX template}
\end{englishabstract}
