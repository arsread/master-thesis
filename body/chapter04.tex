%%==================================================
%% chapter03.tex for SJTU Master Thesis
%% based on CASthesis
%% modified by wei.jianwen@gmail.com
%% Encoding: UTF-8
%%==================================================

\chapter{Java并行框架的功能优化实现}

前文介绍了Deferred Method并行框架。该框架提供了一套易用性高的接口给用户,允许用户在高层次上实现一套自己的代码注入方案。但该方案事实上仍存在着一些不足,例如:

\begin{itemize}
	\item 时序的问题。由于同步处理的方法是单线程执行,故而时序上是可以得到保证的,但这种方法并不是严格并行的;采用基于线程池的异步处理策略和自适应的策略保证了多线程的并行性,但在时序上会产生问题。
	\item 处理策略的问题。该框架只提供了三种策略,这三个策略只是给予用户通过简单地从可用核心的数量上来区分和决定需要使用的策略的选择的,而没有其他的处理策略能通过其他的情况来考虑,例如通过原程序的线程使用情况。
	\item 提交缓存的问题。该框架只有在缓存满的情况下才会自动提交缓存,但事实上有的时候会出现一些情况,使得缓存长时间无法被填满,此时缓存中所储存的方法无法被及时地处理,浪费了空闲的计算资源。
	\item 返回值的问题。该框架只能处理没有返回值的方法,有些分析程序需要在特定的时候使用到之前的分析程序的分析结果,这在Deferred Method默认提供的接口里是无法实现的。
\end{itemize}
		
本章将针对这些问题,一一进行阐述,并提出解决方案和设计。

\section{时序问题和检查点(CheckPoint)的使用}

\subsection{概述}

在前文所列举的方法计数的例子中,由于计数器对原程序的执行并没有时序要求,所以分析程序可以乱序执行;但在一些程序分析的例子里,却存在着时序的约束。一个例子是内容调用树的创建。分析程序需要对树进行初始化,然后才开始整棵树的构建的。但在Deferred Method里所提供的几种处理策略里面,却只有同步处理的方法能保证时序。

一个解决方法是给缓存里的每一个方法加入时间戳。储存的时候,把调用顺序的信息也加入其中,这样可以使得部分需要的方法能有序执行。但这种方法一方面为每个方法都加入了调用信息,使得信息太过详细,可能会引入很多不必要的时序,一定程度上也破坏了程序的并行性;另一方面也加大了程序的开销。

在保证最大程度的并行化地情况下尽可能简单、低开销地加入保证时序的系统,本文提出了检查点(CheckPoint)的技术。

\subsection{设计}

检查点(CheckPoint)的设计思想是在原程序中加入一些点作为标记,在该点以前的所有缓存即可以作为该检查点所标记的缓存。对检查点所标记的缓存,检查点机制能提供两个接口:

\begin{lstlisting}[language=Java]
public interface ProcessingCheckPoint {

	public boolean isProcessed();

	public void awaitProcessed() throws InterruptedException;

}
\end{lstlisting}

其中isProcessed()方法返回一个布尔值,用以表示该检查点所标记的所有缓存是否已经处理完成;awaitProcessed()方法则是用以等待所标记的缓存执行完成,如果在调用方法时缓存尚未处理完成,则该进程会陷入阻塞。

检查点的创建方法则是在DeferredEnv环境中加入相关的方法:

\begin{lstlisting}[language=Java]
	public ProcessingCheckPoint createCheckPoint();
\end{lstlisting}

该方法返回一个检查点的类型的实例,用户可以在这之后使用该实例对程序的时序进行处理。

\subsection{实现}

检查点机制的实现借用了之前的时间戳的方式,但在实现上比之轻量,能够引入较少的开销。

首先,对于每个线程,每次创建的缓存都会被指定一个相对于该线程唯一的编号。该机制另外维护了一个哈希表,每个表项的索引是被分析程序的线程,而每个线程所索引的值都指向一个优先队列。该优先队列存储的是目前已经被处理完成的缓存。

其次,对于每个线程,该实现会创建一个计数器,用以标记从开始的第一个缓存(例如将其编号设置为1)起所已经被处理好的连续的缓存。例如缓存1至3以及5都被处理好了,而4没有被处理好,此时的计数器应为3.

将这两个设定结合起来,即可以完成检查点的机制。当缓存被执行完成的时候,处理器会将该缓存的编号通知给DeferredEnv环境,环境即开始通过该编号与计数器和优先队列进行比对,然后对计数器和优先队列的信息进行更新。如在上例中,当缓存4被处理完成后,该编号信息被提交,环境先将其与计数器进行比对,将计数器更新至4,然后更搜索优先队列,将计数器更新至5,并将5从优先队列中出队。由于优先队列是以最小堆的数据结构实现的,所以在检索上具有效率优势。

检查点在创建的时候,其中也会加入当前缓存的编号信息,所以isProcessed()执行的时候,是将检查点的编号与当前计数器的编号进行比对,以确定在该检查点之前的缓存是否已经都被处理完成。

awaitProcessed()方法的实现则稍微复杂一点,采用了Java中的wait()和notify()的方法。当需要等待时,wait()方法被调用,线程陷入阻塞状态,等待处理的完成。环境另外维护了一个数据结构,用以存储所创建的检查点。由于检查点在每个线程里面总是按缓存顺序创建的,故而并不需要进行排序,简单地使用链表的方式进行储存即可。当缓存被处理完成时,环境更新计数器,然后开始从链表中依次取出检查点进行比对,如果发现某些检查点所标记的所有缓存已经被执行完毕,环境即向相关的线程发出唤醒的请求,以让被阻塞的线程继续执行。

需要注意的是,由于检查点的创建是以缓存为单位的,故而在创建检查点的时候,不管当前线程的缓存是否已满,该缓存都必须被处理。由此可以预见到的一点是,当检查点被频繁创建的时候,线程经常需要提交未满的缓存给处理线程进行处理,从而使得缓存无法发挥其预期的最大效用,会带来效率上的降低。

\section{影子线程的引入及使用}

\subsection{概述}

Deferred Method框架默认提供了几种并行处理的机制,如同步处理、基于线程池的异步处理等等。用户可以针对空闲计算资源的多少确定使用的处理机制。如在单核的情况下,可以使用同步处理的机制;在
