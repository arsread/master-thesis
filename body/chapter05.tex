%%==================================================
%% chapter03.tex for SJTU Master Thesis
%% based on CASthesis
%% modified by wei.jianwen@gmail.com
%% Encoding: UTF-8
%%==================================================

\chapter{实验测试}

本章将选取本文对Deferred Method的部分优化的功能,对其的性能进行测试。实验主要使用两套测试系统,分别为:

\begin{itemize}
	\item Java Grande.这是一套有名的Java并行测试用例,主要包含一些并行程序,本章将选取其中的一部分测试,改写成Deferred Method的形式用于实验。虽然这些测试并不是分析程序,但对于本框架能否在并行上达到并行框架的效果,是有参考意义的。
	\item Dacapo。Dacapo也是一套Java的测试框架,本文将在Dacapo上进行方法计数,比较各种功能和策略的实现性和耗费时间。
\end{itemize}

\section{Java Grande实验}

\subsection{Java Grande介绍}

Java Grande\cite{javagrande}是一套测试集,用于测试和比较一些Java的特定的程序集。这些程序集包括使用大数据,大的I/O或者网络或者大容量内存等的程序。这些程序不禁包括科学和工程上的应用,同时也包括诸如经济学上的一些模型模拟等的应用。

该测试集目前是开源的,包括以下几个部分:

\begin{itemize}
	\item 序列化的测试。适用于单处理器处理。
	\item 多线程测试。适用于使用共享内存的多处理器处理。
	\item MPJ测试。适用于使用分布式内存的多处理器处理。
	\item 语言比较测试。该部分是把序列化测试中的测试集重新用C语言写成,用于语言间的比较。
\end{itemize}

其中,根据本文所面向的程序,我们选择多线程测试的部分。

多线程的部分主要分为底层操作部分、内核部分和大规模应用几个部分。这里我们选择内核部分的测试。这些测试包括:

\begin{itemize}
	\item Series。该测试主要计算函数$f(x) = (x + 1)^x$的前N个傅立叶系数。这个测试里计算了很多的超越函数和三角函数 。它最花费时间的计算来源于在傅立叶系数计算上的循环。每次循环里的迭代都是独立的,所以这个测试能够被分配给多个线程进行处理,每个线程负责一部分计算,最后总结合并结果。
	\item LUFact。该测试用于计算一个$N \times N$的线性系统的解。它使用了LU的因式分解和三角函数计算的方法。事实上它是Linpack测试集的Java版。
	\item SOR。该测试采用了100次迭代的逐次超松弛(succussive over-related)方法,对一个$N \times N$的网格进行处理。这个测试包括了一个多次迭代的循环和两个内嵌的循环,这些循环都对这个网格进行遍历。根据更新的规则,每次迭代里,当更新一个节点的信息的时候,相邻节点的值是需要被使用到的。因此这个测试本质上还是串行的。但为了能提高其并行性,测试采用了一个“红-黑”顺序的机制,该机制能够使得只有相邻的节点需要等待同步,而不是采用一个全局锁。
	\item Crypt。这个测试在一个N字节的数组上采用某些算法进行加密和解密的过程。这个测试共有两个循环,每个循环都能被分解为单独的互不影响的部分,故而可以被并行。
	\item SparseMatmult。这个测试使用了一个非结构化的稀疏矩阵,该矩阵储存于一个行压缩的格式和规定的稀疏结构。主要的测试过程是非直接的访问和不规则的内存引用。该测试会使用200次迭代,并使用一个$N \times N$的矩阵。
\end{itemize}

在这些测试中,由于LUFact引入了全局等待和全局锁,所以不适合使用Deferred Method进行并行。其他的测试,我们都将在源代码的情况下将其改写成采用Deferred Method框架的方法,然后对比其结果。

\subsection{实验结果}

这个测试集为每个测试都提供了大、中、小三种测试规模,这里统一采用大规模进行测试
实验结果表\ref{tab:tab0}所示。

\begin{table}[htbp]
	\centering
	\begin{tabular}{c|c|c|c|c|c}
		\hline
		测试用例 & 测试类型 & 实验一(s) & 实验二(s) & 实验三(s) & 平均用时(s) \\
		\hline
		\multirow{2}{*}{Series} & 原程序 & 103.43 & 111.22 & 104.34 & 106.33 \\
		\cline{2-6}
		& 修改后 & 86.97 & 86.98 & 86.89 &86.95 \\
		\hline
		\multirow{2}{*}{Crypt} & 原程序 & 37.01 & 36.05 & 36.99 & 36.70 \\
		\cline{2-6}
		& 修改后 & 12.19 & 11.888 & 12.042 &12.04 \\
		\hline
		\multirow{2}{*}{SparseMatmult} & 原程序 & 3.23 & 3.20 & 3.16 & 3.20 \\
		\cline{2-6}
		& 修改后 & 0.40 & 0.32 & 0.28 & 0.33 \\
		\hline
		\multirow{2}{*}{SOR} & 原程序 & 267.45 & 300.12 & 280.29 & 282.62 \\
		\cline{2-6}
		& 修改后 & 911.12 & 908.11 & 868.43 & 895.89 \\
		\hline
	\end{tabular}
	\bicaption[tab:tab0]{JavaGrande的实验结果}{JavaGrande的实验结果}{Table}{The test result of JavaGrande}
\end{table}

其中每个同样条件下的实验都做了三次,通过表格的数据可以看出,修改之后,采用了Deferred Method的缓存机制的方法基本有效率上的提升。在Crypt和SparseMatmult测试中,由于使用的原程序的处理线程数量较多,此时Defferred Method中自适应处理策略能够动态调整资源分配状况的优势就体现了出来,这也是它优于其他部分并行框架的一点。

在SOR测试中,Defferred Method的表现不好,这是由于该程序本身结构不适合采用缓存机制所带来的。由此可以看出,该框架在提高并行效率上是有一定作用的,但需要依据不同的测试情况而定,否则会影响结果。

\section{Dacapo实验}

\subsection{Dacapo及DiSL介绍}

本节将使用Dacapo的测试用例集作为测试的基准。通过在这套测试用例集上运行方法计数的程序分析,比较使用Deferred Method与不使用的情况下的影响,并对新增的一些功能对程序运行效率的影响进行测试。在方法注入的工具上,本章选择了DiSL作为工具。

\subsubsection{Dacapo}

Dacapo\cite{dacapo}是一个建立在Java语言上的测试集,其中包括了语言编程、内存管理、计算机体系结构等多方面的内容。这个测试集是由一系列开源的、实际应用的程序组成。本章所做的实验采用了其中部分的测试用例,包括:

\begin{itemize}
	\item avrora。它模拟了一些AVR微控制器上的程序。
	\item batik。它以Apache Batik上一些单元测试为基础,产生了一些向量图。
	\item fop。它读入一个XSL-FO文件,分析该文件,并将该文件格式化,最终产生一个PDF文件。
	\item h2。它执行了一个类似JDBCbench的内存测试用例,同时执行了一个类似银行交易模型的程序。
	\item jython。它作为解释器,解释了基于Python的pybench的测试用例。
	\item luindex。它使用lucene来为一系列的文档创建索引,包括莎士比亚的文集等。
	\item lusearch。它同样使用lucene来搜索一些列文档里的关键字。
	\item pmd。它分析一些Java的类,侦测源代码里的一些问题。
	\item sunflow。它用光线追踪的方法来渲染一系列的图片。
	\item xalan。它将一些XML的文件转换为HTML格式的里的关键字。
	\item pmd。它分析一些Java的类,侦测源代码里的一些问题。
	\item sunflow。它用光线追踪的方法来渲染一系列的图片。
	\item xalan。它将一些XML的文件转换为HTML格式的文件。
\end{itemize}

\subsubsection{Dacapo}

DiSL\cite{disl}是瑞士卢加诺大学开发的一个用于Java代码动态注入的工具。它提供了一套高层的接口,允许用户通过比较简单的调用完成对Java代码的修改。它的切点/通知机制来源于诸如AspectJ等面向方面编程模型,并具有以下特点:

\begin{itemize}
	\item 开放切点的模型。
	\item 兼容Java及Java虚拟机。
	\item 内置通知和本地变量的传递。
	\item 对静态和动态上下文信息有效的获取。
	\item 不需要环绕保卫型的通知。
	\item 高字节码覆盖。
\end{itemize}

DiSL运行时,在加载类时会加入用户定义的代码注入行为,再交付给Java虚拟机进行执行,以达到动态注入的效果。

\subsection{实验设计}

图\ref{fig:code16}是本次实验使用的DiSL代码。

\begin{figure}[!htp]
\begin{lstlisting}[language=Java]

public class DiSLClass {

	static final DeferredEnv<Analyze> def = 
	DeferredExecution.createDeferredEnv(Analyze.class, 
	AnalyzeImpl.class, 
	new AdaptiveProc());
	static final Analyze mc = def.getProxy();

	@Before(marker=BodyMarker.class)
	public static void onMethodEntry() {
		mc.addOne();
	}

}

public interface Analyze extends Deferred{
	public void addOne();
}

public class AnalyzeImpl implements Analyze{
	public void addOne(){
		Profile.cnt.addAndGet();
	}
}

public class Profile {
	public static AtomicLong cnt = new AtomicLong();
}
\end{lstlisting}
\bicaption[fig:code16]{函数计数的DiSL代码}{函数计数的DiSL代码}{Table}{The DiSL class of method count}
\end{figure}

该代码中,mark=BodyMarker.class表明了该分析是以函数体为单位的,@Before表示该方法应用于函数体之前的。

这段代码主要进行了一个将方法计数的代码注入被分析程序的过程。实验先运行没有使用Deferred Method的方法计数程序,得出实验结果;然后运行加入了Defferred Method的方法计数程序,对比效率的变化。接着程序在逐一加入本文提出的一些优化功能,对比它们在效率上带来的变化。注意到该过程是没有返回值的,所以带有返回值的功能在原代码的基础上进行了部分修改,代码如图\ref{fig:code17}所示:

\begin{figure}[!htp]
\begin{lstlisting}[language=Java]
public class DiSLClass {
	@ThreadLocal
	public static final int freq = 50;
	@ThreadLocal
	public static int num=0;
	@ThreadLocal
	public static Ret<Integer> val;

	@Before(marker=BodyMarker.class)
	public static void onMethodEntry() {
		num++;
		if (num == freq) {
			val = SimpleAnalysis.getProfile().testRet();

		}
		if (num == freq * 2) {
			try {
				if (val.get(SimpleAnalysis.getEnv()) != 1){
					System.exit(1);

				} else {
					val = null;

				}
			} catch (InterruptedException e) {
				e.printStackTrace();

			}
		}
		SimpleAnalysis.getProfile().onMethodEntry();
	}
}
\end{lstlisting}
\bicaption[fig:code17]{返回值的DiSL代码}{返回值的DiSL代码}{Table}{The DiSL class of method count}
\end{figure}

该段代码与之前代码的不同在于其并行的方法具有返回值,并且每隔50次方法调用即尝试取得返回值。

该实验每个相同条件下的测试都会运行三次,计算使用时间,然后得出平均值进行分析。

\subsection{实验结果}

实验结果分为三部分,第一部分是测试检查点功能的数据,第二部分是测试影子线程功能的数据,第三部分是测试锁优化和返回值功能的数据。

\subsubsection{检查点功能的数据}

检查点功能的测试分为几组,首先是不使用Deferred Method进行并行分析的原程序,其次是使用了原生Deferred Method框架的原程序,然后在其中加入了检查点的功能,做了四组实验,分别为:

\begin{itemize}
	\item 不用检查点。即在并行过程中采用的框架有检查点功能,但并不进行使用。由于检查点功能引入了一些数据结构和处理机制,故而会带来开销,本实验用于对照这部分开销的大小。
	\item 50、100、1000频度的检查点。此处的频度为每隔一定的方法调用数后即使用检查点的功能,如50频度是指没50次方法调用使用一次检查点,依次类推。
\end{itemize}

实验结果如表\ref{tab:tab1}所示。

\begin{table}[!hpb]
	\centering
	\begin{tabular}{c|c|c|c|c|c|c}
		\hline
		测试用例 & 原程序 & 原生并行 & 不用检查点 & 50频度 & 100频度 & 500频度 \\
		\hline
		pmd & 5910.75 & 6477.75 & 8955.38 & 15251.9 & 10999 & 8085.88 \\
		\hline
		avrora & 25708.5 & 21888.9 & 22953.1 & 55394.9 & 42336.6 & 28957.2 \\
		\hline
		batik & 2985.25 & 3343.12 & 3780 & 7311.25 & 5746.5 & 4646.38 \\
		\hline
		fop & 1905.62 & 2569.12 & 3013.25 & 6954.38 & 5259.12 & 3757.25 \\
		\hline
		h2 & 38556.8 & 27864.1 & 39895.2 & 94455.6 & 66861.5 & 44737.8 \\
		\hline
		jython & 21601.6 & 24181.9 & 29322.6 & 77339.6 & 58188.8 & 36009.5 \\
		\hline
		luindex & 3655.38 & 3364.25 & 4637.88 & 11735.6 & 8389.38 & 5450 \\
		\hline
		lusearch & 15693.5 & 12859 & 15410.1 & 38546.4 & 27176 & 18320.4 \\
		\hline
		sunfulow & 66081.9 & 46959 & 48249.2 & 158186 & 107132 & 63192 \\
		\hline
		xalan & 18415.8 & 15559.4 & 16908.4 & 51323.6 & 34231.1 & 21890.9 \\
		\hline
	\end{tabular}
	\bicaption[tab:tab1]{检查点数据的实验结果}{检查点数据的实验结果}{Table}{The test result of CheckPoint}
\end{table}

实验结果表明,使用了并行框架之后,部分测试用例能够得到效率上的提升。由于方法计数是极小颗粒的并行,所以在并行过程中线程间通信的时间其实和串行执行的时间是差不多的,这也解释了一部分测试为什么会比原程序没有很大提升。从这个角度考虑,事实上Deferred Method也达到了降低进程间通信开销的作用,效率上还是比较理想的。

检查点的引入,在完善原框架功能的情况下,也引入了一部分开销。从表格中可以看出,除了个别例子之外,大部分测试用例里加入检查点功能的Deferred Method框架都能表现出与原框架相比差别不大的效率,可见该功能是可行的。

检查点的使用会强行处理当前进程,故而在检查点使用频繁的时候,缓存会有更高概率在未满的时候即被处理,从而降低了框架效率,本实验也证明了这一点,当使用频度增大时,实验的运行时间明显上升。所以开发者在使用这个功能的时候,必须保证尽量只在有必要的时候使用,否则会影响效率。

\subsubsection{影子线程的数据}

本式样中原生的并行采用的是自适应的处理策略,本实验保持其他条件不便,改用影子线程作为处理器。

实验结果如表\ref{tab:tab2}所示。

\begin{table}[htbp]
	\centering
	\begin{tabular}{c|c|c|c}
		\hline
		测试用例 & 原程序 & 原生并行 & 影子线程 \\
		\hline
		pmd & 5910.75 & 6477.75 & 10250.2 \\
		\hline
		avrora & 25708.5 & 21888.9 & 24962.1 \\
		\hline
		batik & 2985.25 & 3343.12 & 3461.88 \\
		\hline
		fop & 1905.62 & 2569.12 & 3445 \\
		\hline
		h2 & 38556.8 & 27864.1 & 37507.4 \\
		\hline
		jython & 21601.6 & 24181.9 & 27292.2 \\
		\hline
		luindex & 3655.38 & 3364.25 & 3652.88 \\
		\hline
		lusearch & 15693.5 & 12859 & 16205.4 \\
		\hline
		sunfulow & 66081.9 & 46959 & 57320.4 \\
		\hline
		xalan & 18415.8 & 15559.4 & 19318.4 \\
		\hline
	\end{tabular}
	\bicaption[tab:tab2]{影子线程的实验结果}{影子线程的实验结果}{Table}{The test result of Shadow Thread}
\end{table}

该结果与自适应处理策略相比并不理想,这一定程度上是由于实验机器计算资源不够所限制。另一方面,由于测试用例本身设计的缘故,也不适合使用影子线程进行处理。

影子线程的一个好处在于对于原程序的每个线程都只分配一个专用的线程进行处理,这样可以保证该线程所产生的缓存都是被按顺序处理的。开发者可以根据程序本身执行的需要挑选影子线程的处理方式进行处理。

\subsubsection{锁优化和返回值的数据}

这两个方法由于采用了字节吗注入的技术,在运行时受到实验机器的条件限制,所以采用了比较小规模的计算数据。而原程序同样运行在小规模上,以进行对照。

本实验的实验结果如表\ref{tab:tab3}所示。

\begin{table}[htbp]
	\centering
	\begin{tabular}{c|c|c|c}
		\hline
		测试用例 & 原程序 & 锁优化 & 返回值 \\
		\hline
		pmd & 459.5 & 214.375 & 502.125 \\
		\hline
		batik & 1827.75 & 1732.75 & 1928.75 \\
		\hline
		fop & 1554.75 & 926.5 & 1470.25 \\
		\hline
		jython & 3264.5 & 3098.12 & 3710.12 \\
		\hline
		luindex & 269.125 & 226.625 & 278.375 \\
		\hline
		lusearch & 2232.5 & 2017.38 & 2451.88 \\
		\hline
		sunfulow & 1938.88 & 2157.12 & 3274.38 \\
		\hline
	\end{tabular}
	\bicaption[tab:tab3]{锁优化和返回值的实验结果}{锁优化和返回值的实验结果}{Table}{The test result of synchronization optimization and return value}
\end{table}

\begin{itemize}
	\item 在锁优化方面,能看出明显采用了优化之后,程序的运行时间都有了显著的提升,这可以体现出实际应用中,由于锁的阻塞导致的Deferred Method框架的效率降低的现象是比较常见的。通过对这种现象进行优化,我们可以得到效率上的提升。
	\item 在返回值方面, 类似检查点的机制,由于会导致阻塞,所以以一定频率(本例中是每隔50个方法一次)使用返回值功能会造成效率的下降。故而与检查点一样,也建议开发者尽量在必要的情况下使用该功能。
\end{itemize}

\section{本章小结}

本章主要通过本文提出的Deferred Method的框架和在其基础上增加的新的功能进行实验设计,并完成了相关的实验,从实验结果上看,本文所提出的功能均在实验上证明了可行性,对该并行框架的设想和功能的性能的预期均在实验中得到了验证。
