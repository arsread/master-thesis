%%==========================
%% chapter01.tex for SJTU Master Thesis
%% based on CASthesis
%% modified by wei.jianwen@gmail.com
%% version: 0.3a
%% Encoding: UTF-8
%% last update: Dec 5th, 2010
%%==================================================

%\bibliographystyle{sjtu2} %[此处用于每章都生产参考文献]
\chapter{绪论}
\label{chap:Intorduction}

程序分析是近年来的热门话题,而其中利用工具搜集和分析程序动态运行时的信息的技术也被广泛采用。另一方面,随着计算机硬件的发展,多核、分布式技术的适用范围和效率也越来越高,并行程序的重要性也逐渐凸显。本文结合了动态程序分析的特点,将动态程序分析与并行化结合起来,提出并开发了一个适用于编写并行的动态分析程序的框架,并通过实验证明了该框架的实用性和效率。

\section{研究背景}

近年来,随着编程技术和研究的深入,越来越多的语言特性被编程者们所开发和采用,多态、反射、动态加载等技术的使用也越来越广泛。而这些技术对采用对传统的静态编译和分析提出了挑战。污点分析\cite{taint01,taint02}和符号执行\cite{symbExec}等静态分析的方法由于分析效率较低,加之难以收集程序动态运行时的一些信息,其局限性逐渐凸显。为了解决这些问题,动态分析的技术被引入。顾名思义,动态分析即是在运行被分析程序的状态下运行分析程序,从而能够有效地减少分析规模,提高分析精度,并能收集动态信息。一些静态分析技术如程序切片\cite{dynSlc}和符号执行\cite{dynSym}也相应地加入了动态分析的内容。

动态程序的分析有效地减少了相对分析时间,但实际在面对重量级分析的时候,整个分析过程所消耗的时间依然很多,尤其是面对一些本身比较庞大、复杂的程序,在算法已经经过优化的情况下仍旧不能有效地提高运行效率的时候,每个测试用例或者被分析程序动辄需要运行几个小时乃至几天,给程序的分析、查错和优化带了了很大的不便。基于该问题,研究者们采取了不同的解决方案。除了在分析方法本身进行改进分析和在算法的效率以及复杂度上作提高之外,也有另一些方案针对分析程序的执行作加速,其中就包括了并行化的方法。

\section{并行程序分析}

并行化是基于多核、分布式技术等多种硬件技术的发明及成熟而发展起来的技术。在传统的串行计算中,指令是顺序串行执行的,即便后续指令没有用到之前的指令的结果,也只能进行等待,造成了时间上的浪费。并行的理念就是将这些并行度大的指令或者任务派发给空闲的计算资源如cpu核心、物理机等进行分开处理,从而提高计算效率。

在程序分析方面,有一些程序所
